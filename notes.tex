\documentclass[10pt, openany]{article}

\usepackage{enumitem} % itemize stuff
\usepackage{listings} % lstlisting
\usepackage{hyperref} % url, href
\usepackage{float}    % For [H] placement
\usepackage{caption}  % Optional: for custom caption settings
\usepackage{booktabs} % for the table 
\usepackage{amsmath}

\author{Alessio Tanzi}

\begin{document}
  \section{Intro}
  Useful resource: \url{https://tenthousandmeters.com/tag/python-behind-the-scenes/}\par
  this document is about learning python. The following is the hello
  world program:
  \begin{lstlisting}
print("Hello World");
  \end{lstlisting}

  \section{Variables}
  Each variable is \textbf{connected} to a value.\par
  Uppercase letters in variable names have special meaning (later)\par
  Internally, variables are \textbf{references} to values in 
  memory.\par
  \subsubsection{Strings}
  You can use both " and ' to delimit them.\par
  Concat them with + to write them over multiple lines, and write 
  \\n to write newline.\par
  The internal representation of strings in python is actually not 
  that simple.\par
  The string has 2 possible states: \textbf{compact} and 
  \textbf{legacy}, in which compact representation basically is a 
  list of UTF-8 characters and is used only \textit{maximum character
  and size are known at creation time} (eg for string literals).\par
  Otherwise, it will revert to the legacy representation, which, 
  depending on the content of the string, can be of 3 \textit{kinds}
  \begin{itemize}[noitemsep, topsep=0pt]
    \item Latin-1
    \item UCS-2
    \item UCS-4
  \end{itemize}
  Reported here is the actual struct used in CPython as of PEP393
  \begin{figure}[H]
  \captionsetup{labelformat=empty}  % Suppress the caption label
  \begin{lstlisting}
typedef struct {
  PyObject_HEAD
  Py_ssize_t length;
  Py_hash_t hash;
  struct {
      unsigned int interned:2;
      unsigned int kind:2;
      unsigned int compact:1;
      unsigned int ascii:1;
      unsigned int ready:1;
  } state;
  wchar_t *wstr;
} PyASCIIObject;

typedef struct {
  PyASCIIObject _base;
  Py_ssize_t utf8_length;
  char *utf8;
  Py_ssize_t wstr_length;
} PyCompactUnicodeObject;

typedef struct {
  PyCompactUnicodeObject _base;
  union {
      void *any;
      Py_UCS1 *latin1;
      Py_UCS2 *ucs2;
      Py_UCS4 *ucs4;
  } data;
} PyUnicodeObject;
  \end{lstlisting}
  \end{figure}
  link to the documentation: \url{https://peps.python.org/pep-0393/#string-creation}\par
  
  We have methods to manipulate the string, like \texttt{strip}, 
  \texttt{find}, \texttt(index), \texttt{split}, \texttt{join},
  we can \textbf{use all comparisons operations lexicograhycal},
  \par
  We can also query for membership like
  \begin{lstlisting}
    'a' in 'apple' == True
  \end{lstlisting}
  
  \section{Numbers}
  There are 3 types of number in python: \textbf{integers}, 
  \textbf{floating-point numbers} and \textbf{complex numbers}. The 
  standard library also gives us \texttt{decimal.Decimal} and 
  \texttt{fractions.Fraction}.\par
  To create a complex number, just append the 'j' to a numeric literal
  \par
  \begin{lstlisting}
inum = -32432
fnum = 3.32423
cnum = 3.14 - 1j
  \end{lstlisting}
\textbf{Integers} in python are \textbf{arbitrary-precision integers}.
  \begin{lstlisting}
typedef struct {
    PyObject ob_base;
    Py_ssize_t ob_size; /* Number of items in variable part */
} PyVarObject;

struct _longobject {
    PyVarObject ob_base; // expansion of PyObject_VAR_HEAD macro
    digit ob_digit[1];
};
  \end{lstlisting}
  the \texttt{ob\_digit} member is a pointer to an array of digits.
  More information on this bignum arithmetic implementation 
  \url{https://tenthousandmeters.com/blog/python-behind-the-scenes-8-how-python-integers-work/}\par
  This comes with performance implications for each integer 
  operation and the memory consumption of each integer, which is 
  proportional to the number itself. For reference \textbf{small 
  numbers take 28 bytes}. You can verify that by calling the 
  \texttt{bit\_length} method on an integer
  \begin{itemize}[noitemsep, topsep=0pt]
    \item a reference count \texttt{ob\_refcnt}: 8 bytes
    \item a type \texttt{ob\_type}: 8 bytes
    \item an object's size \texttt{ob\_size}: 8 bytes
    \item \texttt{ob\_digit}: 4 bytes.
  \end{itemize}
  \textbf{Floating} numbers are instead double precision floatin point
  numbers, stored in a \texttt{PyObject} type, which is a reference
  counted object.
  \begin{lstlisting}
typedef struct { 
    PyObject_HEAD
    double ob_fval; 
} PyFloatObject
  \end{lstlisting}
  \textbf{Complex} numbers are basically a pair of floating point 
  numbers (double precision)
  \begin{lstlisting}
typedef struct {
    PyObject_HEAD
    double cval_real;   // Real part
    double cval_imag;   // Imaginary part
} PyComplexObject;
  \end{lstlisting}
  For each of the number types the the following operations 
  are defined
  \begin{table}[ht]
    \centering
    \caption{Built-in python numbers operations}
    \begin{tabular}{@{}lll@{}}
        \toprule
        \texttt{Operation}         & \texttt{Description}                                  & \texttt{Notes}           \\ \midrule
        \texttt{x + y}        & Sum of \(x\) and \(y\)                            &                  \\
        \texttt{x - y}        & Difference of \(x\) and \(y\)                     &                  \\
        \texttt{x * y}        & Product of \(x\) and \(y\)                        &                  \\
        \texttt{x / y}        & Quotient of \(x\) and \(y\)                       &                  \\
        \texttt{x // y}       & Floored quotient of \(x\) and \(y\)               & (1)(2)          \\
        \texttt{x \% y}       & Remainder of \(x / y\)                            & (2)              \\
        \texttt{-x}           & \(x\) negated                                      &                  \\
        \texttt{+x}           & \(x\) unchanged                                    &                  \\
        \texttt{abs(x)}       & Absolute value or magnitude of \(x\)               &                  \\
        \texttt{int(x)}       & \(x\) converted to integer                          & (3)(6)          \\
        \texttt{float(x)}     & \(x\) converted to floating point                   & (4)(6)          \\
        \texttt{complex(re, im)} & A complex number with real part \(re\), imaginary part \(im\) (defaults to zero) & (6) \\
        \texttt{c.conjugate()} & Conjugate of the complex number \(c\)              &                  \\
        \texttt{divmod(x, y)} & The pair \((x / y, x \% y)\)                     & (2)              \\
        \texttt{pow(x, y)}    & \(x\) to the power \(y\)                           & (5)              \\
        \texttt{x ** y}       & \(x\) to the power \(y\)                           & (5)              \\ 
        \bottomrule
    \end{tabular}
    \label{tab:math_operations}
  \end{table}
  Furthermore, integers also feature \textbf{bitwise operations}, 
  which are \texttt{|} (or), \texttt{\^} (xor), 
  \texttt{\&} (and), 
  \texttt{<<} (left shift), \texttt{>>} (right shift), 
  \texttt{~} (not) \par

  \section{Boolean}
  The \texttt{bool} type has 2 possible values: \texttt{True} and 
  \texttt{False}, with a constructor

  \section{Containers Intro}
  Python supports 3 types of containers: \textbf{sequences} and 
  \textbf{set types} and \textbf{mapping objects}, each with its
  interface, which is meant to be implemented if you want to write 
  your custom container.\par
  A method which is shared by all of them is the one that returns an
  interator over the container (part of the \textbf{iterable} 
  interface)
  \begin{lstlisting}
container.__iter__()
  \end{lstlisting}
  While an \textbf{iterator} implmenets
  \begin{lstlisting}
iterator.__iter__()
iterator.__next__() # raise StopIteration exception at end
  \end{lstlisting}
  Example of a first custom container (with immutable elements, 
  hence useless, for the sake of using iterators)
  \begin{lstlisting}
def ordinalStr(num):
  match num:
    case 1:
      return '1st'
    case 2:
      return '2nd'
    case 3:
      return '3rd'
    case _:
      return str(num) + 'th'

# trying out iterators for myself
class Thing:
  # private method starts with __

  # private variable
  _numbers = [ordinalStr(x) for x in range(1, 10)]

  # constructor
  def __init__(self, x):
    # another variable
    self._firstStr = self._numbers[0]
    self._arg = x

  # iterable interface
  def __iter__(self):
    return self._numbers.__iter__()

  # to print it directly
  def __repr__(self):
    className = type(self).__name__
    return f'{className}()'

  def getArg(self):
    return self._arg;
  \end{lstlisting}
  Starting to see \textbf{sequences}

  \section{Sequence Types}
  There are 3: \texttt{\textbf{list}}, 
  \texttt{\textbf{tuple}}, \texttt{\textbf{range}}\par

  All sequences have common operations, eg. membership, indexing, 
  concatenation, repetition, minmax, count.\par
  Of course, these operators work if you defined the interface, in
  particular in min/max, you need to define the \textbf{comparison}
  operations (\texttt{\_\_lt\_\_}, \texttt{\_\_le\_\_}, 
  \texttt{\_\_eq\_\_}, \texttt{\_\_ne\_\_}, \texttt{\_\_gt\_\_}, 
  \texttt{\_\_ge\_\_})
  \begin{table}[t]
    \centering
    \caption{Sequence operations}
    \begin{tabular}{@{}ll@{}}
      \toprule
      \textbf{Operation} & \textbf{Outcome} \\
      \midrule
      \texttt{x in s} & \texttt{True} if an item of s is equal to x,
        else \texttt{False} \\
      \texttt{x not in x} & \texttt{False} if an item of s is equal
        to x, else \texttt{True} \\
      \texttt{s + t} & the concatenation of s and t \\
      \texttt{s * n} or \texttt{n * s} & equivalent to adding s to
        itself n times \textbf{shallow copy}\\
      \texttt{s[i]} & ith element of s \\
      \texttt{s[i:j]} & slice of s from i to j \\
      \texttt{s[i:j:k]} & slice of s from i to j with step \\
      \texttt{len(s)} & length of s \\
      \texttt{min(s)} & smallest element of s \\
      \texttt{max(s)} & largest element of s \\
      \texttt{s.index(x[, i[, j]])} & find first occurrence of x,
        optionally within a subsequence from i to j \\
      \texttt{s.count(x)} & number of occurrences of x in s \\
      \bottomrule
    \end{tabular}
  \end{table}
  If you give a \textbf{negative index} \texttt{i}, then the actual
  formula used to index the object is \texttt{len(s) + i}\par
  Note that \texttt{str} objects are immutable (like strings in java)
  and the equivalent of string builder is 
  \textbf{\texttt{io.StringIO}} (or use \texttt{str.join()} on an 
  \textit(iterable))\par
  \textbf{\texttt{tuple} and \texttt{range} are immutable}, hence
  \textbf{hashable}, which means that they can be used as key types
  in an mapping object like \texttt{dict}.\par
  \textbf{Mutable} sequence types like \texttt{list} support 
  additional operations, i.e. assignment, deletion, insertion, 
  extension.
  \begin{table}[t]
    \centering
    \caption{Mutable Sequence Operations}
  \end{table}
\end{document}
